% Options for packages loaded elsewhere
\PassOptionsToPackage{unicode}{hyperref}
\PassOptionsToPackage{hyphens}{url}
\PassOptionsToPackage{dvipsnames,svgnames,x11names}{xcolor}
%
\documentclass[
]{agujournal2019}

\usepackage{amsmath,amssymb}
\usepackage{iftex}
\ifPDFTeX
  \usepackage[T1]{fontenc}
  \usepackage[utf8]{inputenc}
  \usepackage{textcomp} % provide euro and other symbols
\else % if luatex or xetex
  \usepackage{unicode-math}
  \defaultfontfeatures{Scale=MatchLowercase}
  \defaultfontfeatures[\rmfamily]{Ligatures=TeX,Scale=1}
\fi
\usepackage{lmodern}
\ifPDFTeX\else  
    % xetex/luatex font selection
\fi
% Use upquote if available, for straight quotes in verbatim environments
\IfFileExists{upquote.sty}{\usepackage{upquote}}{}
\IfFileExists{microtype.sty}{% use microtype if available
  \usepackage[]{microtype}
  \UseMicrotypeSet[protrusion]{basicmath} % disable protrusion for tt fonts
}{}
\makeatletter
\@ifundefined{KOMAClassName}{% if non-KOMA class
  \IfFileExists{parskip.sty}{%
    \usepackage{parskip}
  }{% else
    \setlength{\parindent}{0pt}
    \setlength{\parskip}{6pt plus 2pt minus 1pt}}
}{% if KOMA class
  \KOMAoptions{parskip=half}}
\makeatother
\usepackage{xcolor}
\setlength{\emergencystretch}{3em} % prevent overfull lines
\setcounter{secnumdepth}{5}
% Make \paragraph and \subparagraph free-standing
\makeatletter
\ifx\paragraph\undefined\else
  \let\oldparagraph\paragraph
  \renewcommand{\paragraph}{
    \@ifstar
      \xxxParagraphStar
      \xxxParagraphNoStar
  }
  \newcommand{\xxxParagraphStar}[1]{\oldparagraph*{#1}\mbox{}}
  \newcommand{\xxxParagraphNoStar}[1]{\oldparagraph{#1}\mbox{}}
\fi
\ifx\subparagraph\undefined\else
  \let\oldsubparagraph\subparagraph
  \renewcommand{\subparagraph}{
    \@ifstar
      \xxxSubParagraphStar
      \xxxSubParagraphNoStar
  }
  \newcommand{\xxxSubParagraphStar}[1]{\oldsubparagraph*{#1}\mbox{}}
  \newcommand{\xxxSubParagraphNoStar}[1]{\oldsubparagraph{#1}\mbox{}}
\fi
\makeatother


\providecommand{\tightlist}{%
  \setlength{\itemsep}{0pt}\setlength{\parskip}{0pt}}\usepackage{longtable,booktabs,array}
\usepackage{calc} % for calculating minipage widths
% Correct order of tables after \paragraph or \subparagraph
\usepackage{etoolbox}
\makeatletter
\patchcmd\longtable{\par}{\if@noskipsec\mbox{}\fi\par}{}{}
\makeatother
% Allow footnotes in longtable head/foot
\IfFileExists{footnotehyper.sty}{\usepackage{footnotehyper}}{\usepackage{footnote}}
\makesavenoteenv{longtable}
\usepackage{graphicx}
\makeatletter
\newsavebox\pandoc@box
\newcommand*\pandocbounded[1]{% scales image to fit in text height/width
  \sbox\pandoc@box{#1}%
  \Gscale@div\@tempa{\textheight}{\dimexpr\ht\pandoc@box+\dp\pandoc@box\relax}%
  \Gscale@div\@tempb{\linewidth}{\wd\pandoc@box}%
  \ifdim\@tempb\p@<\@tempa\p@\let\@tempa\@tempb\fi% select the smaller of both
  \ifdim\@tempa\p@<\p@\scalebox{\@tempa}{\usebox\pandoc@box}%
  \else\usebox{\pandoc@box}%
  \fi%
}
% Set default figure placement to htbp
\def\fps@figure{htbp}
\makeatother
% definitions for citeproc citations
\NewDocumentCommand\citeproctext{}{}
\NewDocumentCommand\citeproc{mm}{%
  \begingroup\def\citeproctext{#2}\cite{#1}\endgroup}
\makeatletter
 % allow citations to break across lines
 \let\@cite@ofmt\@firstofone
 % avoid brackets around text for \cite:
 \def\@biblabel#1{}
 \def\@cite#1#2{{#1\if@tempswa , #2\fi}}
\makeatother
\newlength{\cslhangindent}
\setlength{\cslhangindent}{1.5em}
\newlength{\csllabelwidth}
\setlength{\csllabelwidth}{3em}
\newenvironment{CSLReferences}[2] % #1 hanging-indent, #2 entry-spacing
 {\begin{list}{}{%
  \setlength{\itemindent}{0pt}
  \setlength{\leftmargin}{0pt}
  \setlength{\parsep}{0pt}
  % turn on hanging indent if param 1 is 1
  \ifodd #1
   \setlength{\leftmargin}{\cslhangindent}
   \setlength{\itemindent}{-1\cslhangindent}
  \fi
  % set entry spacing
  \setlength{\itemsep}{#2\baselineskip}}}
 {\end{list}}
\usepackage{calc}
\newcommand{\CSLBlock}[1]{\hfill\break\parbox[t]{\linewidth}{\strut\ignorespaces#1\strut}}
\newcommand{\CSLLeftMargin}[1]{\parbox[t]{\csllabelwidth}{\strut#1\strut}}
\newcommand{\CSLRightInline}[1]{\parbox[t]{\linewidth - \csllabelwidth}{\strut#1\strut}}
\newcommand{\CSLIndent}[1]{\hspace{\cslhangindent}#1}

\usepackage{url} %this package should fix any errors with URLs in refs.
\usepackage{lineno}
\usepackage[inline]{trackchanges} %for better track changes. finalnew option will compile document with changes incorporated.
\usepackage{soul}
\linenumbers
\makeatletter
\@ifpackageloaded{caption}{}{\usepackage{caption}}
\AtBeginDocument{%
\ifdefined\contentsname
  \renewcommand*\contentsname{Table of contents}
\else
  \newcommand\contentsname{Table of contents}
\fi
\ifdefined\listfigurename
  \renewcommand*\listfigurename{List of Figures}
\else
  \newcommand\listfigurename{List of Figures}
\fi
\ifdefined\listtablename
  \renewcommand*\listtablename{List of Tables}
\else
  \newcommand\listtablename{List of Tables}
\fi
\ifdefined\figurename
  \renewcommand*\figurename{Figure}
\else
  \newcommand\figurename{Figure}
\fi
\ifdefined\tablename
  \renewcommand*\tablename{Table}
\else
  \newcommand\tablename{Table}
\fi
}
\@ifpackageloaded{float}{}{\usepackage{float}}
\floatstyle{ruled}
\@ifundefined{c@chapter}{\newfloat{codelisting}{h}{lop}}{\newfloat{codelisting}{h}{lop}[chapter]}
\floatname{codelisting}{Listing}
\newcommand*\listoflistings{\listof{codelisting}{List of Listings}}
\makeatother
\makeatletter
\makeatother
\makeatletter
\@ifpackageloaded{caption}{}{\usepackage{caption}}
\@ifpackageloaded{subcaption}{}{\usepackage{subcaption}}
\makeatother

\usepackage{bookmark}

\IfFileExists{xurl.sty}{\usepackage{xurl}}{} % add URL line breaks if available
\urlstyle{same} % disable monospaced font for URLs
\hypersetup{
  pdftitle={Shifts in Climate Change and Their Influence on Deep-Time Fossil Composition Dynamics},
  pdfauthor={Gabriel Munoz-Acevedo; JP Lessard},
  pdfkeywords={paleoclimates, biotas, historical biogeography},
  colorlinks=true,
  linkcolor={blue},
  filecolor={Maroon},
  citecolor={Blue},
  urlcolor={Blue},
  pdfcreator={LaTeX via pandoc}}



\draftfalse

\begin{document}
\title{Shifts in Climate Change and Their Influence on Deep-Time Fossil
Composition Dynamics}

\authors{Gabriel Munoz-Acevedo\affil{1}, JP Lessard\affil{1}}
\affiliation{1}{Concordia University, }



\begin{abstract}
Novel environmental pressures influences community assembly dynamics and
broad scale patterns in biodiversity composition. While modern community
assembly theory offers insights into the effects of climate change on
broad-scale biodiversity patterns, these are often based on short-term
snapshots, limiting our understanding of long-term processes underlying
patterns of species turnover across ecological communities. Morevoer, a
critical gap remains in elucidating how deep-time climatic shifts have
shaped historical community assembly, as represented by the fossil
record. In this study, we aim to uncover the historical dynamics into
the mechanisms that could shape present and future biodiversity patterns
in response to ongoing climate change. We evaluate two hypotheses
regarding the impact of paleo-climate change on shifts in fossil species
composition between distinct sites within a broader region, also known
as beta-diversity. The Severity of Climate Change hypothesis posits that
species turnover increases with the magnitude of climate change, while
the Climatic Instability hypothesis suggests that frequent climate
fluctuations promote higher species turnover through dynamic niche
shifts. We evaluate our hypothesis alongside the influences of
contemporary niche and dispersal assembly processes, which are typically
considered in current ecological studies. To test these, we integrate
fossil and paleoclimatic data focusing on large-mammals in the Neogene,
a time marked by major climatic shifts in Northern latitudes. We found
that \ldots{}
\end{abstract}





\section{Introduction}\label{introduction}

The impact of climate change on biodiversity patterns, particularly at
broad spatial scales, has been a central focus in modern community
assembly theory (Ackerly, 2003; Harrison et al., 2020; Kraft et al.,
2015). However, much of the current research remains limited by the
temporal scope of available data. Ecologists critic community assembly
studies on their relying on ``one-time snapshots'' of species
distributions to infer long-term processes (Chang \& HilleRisLambers,
2016; Harrison et al., 2020; Khattar et al., 2021). Such approach als
falls short in capturing how species turnover is shaped by climate
dynamics over extended geological periods (DiMichele et al., 2004;
Fukami, 2015; Weiher et al., 2011). The result is a fragmented
understanding of how long-term climate change dynamics has influenced
species turnover across large spatial scales, especially over timescales
relevant to both evolutionary and ecological community assembly
processes (Erwin, 2009; J. B. Jackson \& Erwin, 2006; S. T. Jackson \&
Blois, 2015).

In addition to the typically evaluated contemporary niche and dispersal
based processes, two hypotheses offer a framework to understand climate
change's impact on biodiversity distribution across several time
periods. The~severity of climate change~hypothesis suggests that if the
magnitude of climate change drives turnover, then the compositional
differences between fossil assemblages in space and over time will
increase with larger climatic magnitude deltas. This is because shifts
in the average climate conditions of a region over multi-million years
may encourage species origination through local adaptation or migration
from neighboring regions, while also increasing global extinction and
local extirpation rates (Davis et al., 2005; Davis \& Shaw, 2001; Fuente
et al., 2022; Vasconcelos et al., 2022). The~Climatic
Instability~hypothesis posits that compositional differences between
fossil assemblages in space to be associated with the degree of temporal
change in a region's climate spatial heterogeneity. This is because
unstable climates can create new ecological niches within a region that
offer opportunities for diversification and extinction, while stable
climates lessen the pressures for adaptation, migration, and extinction
(Cavender-Bares et al., 2016; Gerhold et al., 2018; Sonne et al., 2022;
Vasconcelos et al., 2022)

Along Earth's history, broad scale climatic changes (Zachos et al.,
2001) create novel pressures that influences species' distributional
ranges~(Alroy et al., 2000; Erwin, 2009; Nogués-Bravo, 2009;
Nogués-Bravo et al., 2018). Thus far, we understand very little the role
of these deep time climatic changes in shaping macroscale patterns in
biodiversity (Bartlein \& Prentice, 1989; Benson et al., 2021; Myers et
al., 2015). This knowledge can be critical to make informed predictions
about past, current, and future biodiversity distribution in light of
ongoing climatic changes (Hagen, 2023; Kiessling et al., 2023;
Nogués-Bravo et al., 2018). Paleoclimate models---reconstructions of
Earth's climate over geological timescales---offer a promising avenue
for integration with independent paleoecological data (Barr, 2017; Faith
\& Behrensmeyer, 2013; Hargreaves \& Annan, 2009; Ludwig et al., 2019).
By combining paleoclimatic reconstructions, fossil data, and community
assembly theory, we can model a relationship between historical climatic
changes and the changes the compositional dissimilarity of fossil
assemblages at macroecological scales across stratigraphic periods
(Blanco et al., 2021; Jabot et al., 2020; Mottl et al., 2021;
\textbf{naglu2023across?}). Here we aim to model the compositional
dissimilarity of large-mammal fossil assemblages within North America
and Europe and during the Neogene period. Specifically, we ask 1) Does
the magnitude of climate change influence fossil assemblage
dissimilarity? and 2) To what extent do temproal instability and spatial
heterogeneity of climate contribute to fossil assemblage dissimilarity?
While more extreme climatic shifts are expected to increase turnover
through mechanisms such as origination, extirpation, and extinction,
their impact on assemblage dissimilarity will be amplified if these
shifts are also accompanied by significant climatic instability and
fluctuations in spatial heterogeneity.

\section{Methods}\label{methods}

\subsection{Data}\label{data}

\subsubsection{Study taxa}\label{study-taxa}

Fossils of large mammals are particularly useful to uncover
macro-evolutionary trends as they are often preserved with higher
spatial fidelity and therefore less prone to taphonomic biases compared
to other fossil organisms. Moreover, the results of centuries of
paleontological research are now digitally available in aggregated
fossil databases with standardized taxonomic nomenclature and geotagged
records. For this study, we gathered fossil data from the NOW database,
excluding all records where the genus or species was classified as
indeterminate (``indet.'').

\subsubsection{Temporal scope}\label{temporal-scope}

The Neogene period (23-2.6 million years ago) provides an ideal case
study to test these hypotheses, as this period was characterized by
significant climatic changes. This period saw the emergence of many
modern plant and animal families, as well as the extinction of many
others that were widespread. During this time, the Earth's climate
transitioned from a greenhouse to an icehouse state, with the onset of
the Northern Hemisphere glaciation around 3 million years ago. These led
to the transformation of tropical areas in high latitudes into open
landscapes such as savannas and prairies, and the expansion of temperate
and boreal forests. These changes were driven by a combination of
tectonic and oceanographic events, including the establishment of the
modern-day Gulf Stream, the uplift of the Rocky Mountains and the Alps.

The Neogene is divided into several stratigraphic stages based on
geologic time. We aggregated fossil observations into categorical
stratigraphic ages based on their mid-point betwen the maximun and
minumum radiocarbon based age estimates. The \textbf{Aquitanian} (23--21
Ma) marks the beginning of the Neogene and is characterized by early
Miocene climate conditions. The \textbf{Burdigalian} (20--16 Ma)
follows, marked by warmer climates, while the \textbf{Langhian} (15--14
Ma) and \textbf{Serravallian} (13--12 Ma) represent mid-Miocene periods
of global cooling. The late Miocene stages, \textbf{Tortonian} (11--8
Ma) and \textbf{Messinian} (7--5 Ma), involve further cooling and
tectonic shifts, culminating in the Mediterranean Salinity Crisis during
the Messinian. The Pliocene stages, \textbf{Zanclean} (5--4 Ma) and
\textbf{Piacenzian} (3--2 Ma), feature continued global cooling, setting
the stage for the Pleistocene glaciations.

\subsubsection{Region of interest}\label{region-of-interest}

The study area spans North-America and Europe. We divided continental
masses into subregions: Eastern, Central, and Western USA, as well as
Western Europe, Eastern Europe, and the Caucasus region. These
geographic divisions capture key continental-scale gradients, allowing
for analysis of species turnover and the regional variability in
climatic effects. We aggregated present-day geographic coordinates of
fossil observations using an equal-area hexagonal grid (100 Km spacing)
rotated to Phanerozoic stratigraphic stages (Merdith et al., 2021),
aggregating all coordinates within the same grid cell, implemented in
the \texttt{grid} method of the \texttt{paelorotate} function of the
\texttt{paleoverse} (Jones et al., 2023) package for \texttt{R}.
Further, we aggregated fossil' hexagonal binned records into
square-gridded fossil paleocommunities, each of of 25 degree squared
area. We computed sampling effort per grid as the count of all unique
Locality Identification Numbers (LIDNUM), a unique identifier assigned
to each fossil locality represented in the NOW database.

\subsubsection{Palaeoatmospheric data}\label{palaeoatmospheric-data}

We utilized paleo-atmospheric temperature reconstructions for the
Northern Hemisphere (Hagen et al., 2019) as a proxy for macro-scale
climatic variation across continental regions and Neogene time periods.
The dataset from Hagen et al.~(2019) spans the Cenozoic (60 Ma to the
present), and we extracted a subset corresponding to the temporal range
of our fossil data. To align the spatial resolution of the climate data
with that of our fossil paleocommunities (5x5 degree grid cells), we
aggregated the 1x1 degree grid cell temperature records by calculating
simple means within each 5x5 degree grid cell.

To assess the distribution of temperature across regions within each
stratigraphic stage, we aggregated 1 Ma temperature records by
calculating simple means. To quantify the magnitude of climate change,
we computed the temporal slope for each grid cell by fitting a linear
regression model to the 1 Ma temperature records within each
stratigraphic stage as a function of time. The slope of this model
represents the rate of temperature change, while the standard deviation
of the residuals from the regression captures temporal variability in
climate for each region within a given stratigraphic stage.

\subsection{Statistical analyses}\label{statistical-analyses}

\subsubsection{Broad-scale fossil diversity
patterns}\label{broad-scale-fossil-diversity-patterns}

\paragraph{Taxonomic richness}\label{taxonomic-richness}

Our analysis began with quantifying the relative abundance of fossil
genera across various grid cells and stratigraphic stages. For each grid
cell, we normalized the genus counts by dividing the number of
occurrences of each genus by the total species count at that site. This
proportional representation enabled meaningful comparisons of genus
composition across both spatial and temporal scales.

\paragraph{Paleocommunity composition}\label{paleocommunity-composition}

To evaluate the ecological dissimilarity between regions and time
periods, we constructed genus-by-site matrices for each stratigraphic
stage. We created these matrices by cross-tabulating the normalized
genus abundances against the sites, with the exclusion of any sites
lacking fossil records (i.e., those with zero total counts), ensuring
our analysis was restricted to regions with available data (at least 3
unique genera recorded). We then calculated pairwise Bray-Curtis
dissimilarities between grid cells within each stratigraphic stage. This
approach provided an overview of taxonomic turnover of continental
fossil communities across different stratigraphic stages.

\subsubsection{Paleoclimatic variables}\label{paleoclimatic-variables}

From Hagen's (2019) 1-million-year (1Ma) resolution paleoclimatic
reconstructions, we derived a set of four paleoclimatic variables, each
aggregated by stratigraphic period within the Neogene. Specifically,
these variables include: 1) \textbf{Average atmospheric temperature
(t\_var)}, calculated as the mean temperature across all 1Ma intervals
within each stratigraphic stage; 2) \textbf{Magnitude of atmospheric
temperature change (t\_slope)}, which represents the slope of a linear
regression model fitted to the relationship between age and temperature
for a given stratigraphic stage; 3) \textbf{Atmospheric temperature
temporal instability (t\_inst)}, measured as the standard deviation of
the residuals from the linear model used to estimate t\_slope; and 4)
\textbf{Spatial heterogeneity of atmospheric temperature (t\_svar)},
computed as the mean difference between a focal temperature cell and its
eight neighboring cells within a stratigraphic period, following the
Horn algorithm.

\paragraph{Fluctuations in spatial heterogeneity of atmospheric
temperature
(t\_svar)}\label{fluctuations-in-spatial-heterogeneity-of-atmospheric-temperature-t_svar}

\subsubsection{General dissimilarity
modelling}\label{general-dissimilarity-modelling}

In this study, we applied General Dissimilarity Models (GDM) to explore
spatial variation in fossil paleocommunity composition across
environmental gradients and stratigraphic stages. GDMs offer a
non-linear framework to quantify species turnover across environmental
gradients while accounting for spatial effects on ecological
dissimilarity. By focusing our analysis on fossil genera instead of
species, we minimized biases associated with inconsistent taxonomic
resolution across fossil sites. Separate GDMs were fitted to
compositional dissimilarity matrices for each stratigraphic stage,
modeling compositional turnover as a function of four paleoclimatic
temperature variables: average atmospheric temperature (t\_mean),
magnitude of temperature change (t\_slope), temporal temperature
instability (t\_var), and spatial temperature heterogeneity (t\_svar).
The geographic distance between fossil paleocommunities was included as
a covariate, calculated as the Euclidean distance between site centroids
within each paleocommunity grid. The relative contribution of each
variable was determined through iterative adjustment of spline functions
to maximize the fit between observed and predicted dissimilarity. We
then examined how the explanatory power of each temperature variable
shifted across stratigraphic stages. Lastly, by transforming and
rescaling each environmental predictor to a common scale of biological
relevance, we mapped geographic variation in community assembly
processes over time, enabling a more meaningful comparison of climatic
drivers of species turnover across regions and stages.

\section{Results}\label{results}

\section{Discussion}\label{discussion}

\subsubsection{}\label{section}

\subsubsection*{}\label{section-1}
\addcontentsline{toc}{subsubsection}{}

\phantomsection\label{refs}
\begin{CSLReferences}{1}{0}
\bibitem[\citeproctext]{ref-ackerly2003community}
Ackerly, D. D. (2003). Community assembly, niche conservatism, and
adaptive evolution in changing environments. \emph{International Journal
of Plant Sciences}, \emph{164}(S3), S165--S184.

\bibitem[\citeproctext]{ref-alroy2000global}
Alroy, J., Koch, P. L., \& Zachos, J. C. (2000). Global climate change
and north american mammalian evolution. \emph{Paleobiology},
\emph{26}(S4), 259--288.

\bibitem[\citeproctext]{ref-barr2017signal}
Barr, W. A. (2017). Signal or noise? A null model method for evaluating
the significance of turnover pulses. \emph{Paleobiology}, \emph{43}(4),
656--666.

\bibitem[\citeproctext]{ref-bartlein1989orbital}
Bartlein, P. J., \& Prentice, I. (1989). Orbital variations, climate and
paleoecology. \emph{Trends in Ecology \& Evolution}, \emph{4}(7),
195--199.

\bibitem[\citeproctext]{ref-benson2021biodiversity}
Benson, R. B., Butler, R., Close, R. A., Saupe, E., \& Rabosky, D. L.
(2021). Biodiversity across space and time in the fossil record.
\emph{Current Biology}, \emph{31}(19), R1225--R1236.

\bibitem[\citeproctext]{ref-blanco2021punctuated}
Blanco, F., Calatayud, J., Martı́n-Perea, D. M., Domingo, M. S.,
Menéndez, I., Müller, J., et al. (2021). Punctuated ecological
equilibrium in mammal communities over evolutionary time scales.
\emph{Science}, \emph{372}(6539), 300--303.

\bibitem[\citeproctext]{ref-cavender2016evolutionary}
Cavender-Bares, J., Ackerly, D. D., Hobbie, S. E., \& Townsend, P. A.
(2016). Evolutionary legacy effects on ecosystems: Biogeographic
origins, plant traits, and implications for management in the era of
global change. \emph{Annual Review of Ecology, Evolution, and
Systematics}, \emph{47}, 433--462.

\bibitem[\citeproctext]{ref-chang2016integrating}
Chang, C., \& HilleRisLambers, J. (2016). Integrating succession and
community assembly perspectives. \emph{F1000Research}, \emph{5}.

\bibitem[\citeproctext]{ref-davis2001range}
Davis, M. B., \& Shaw, R. G. (2001). Range shifts and adaptive responses
to quaternary climate change. \emph{Science}, \emph{292}(5517),
673--679.

\bibitem[\citeproctext]{ref-davis2005evolutionary}
Davis, M. B., Shaw, R. G., \& Etterson, J. R. (2005). Evolutionary
responses to changing climate. \emph{Ecology}, \emph{86}(7), 1704--1714.

\bibitem[\citeproctext]{ref-dimichele2004long}
DiMichele, W. A., Behrensmeyer, A. K., Olszewski, T., Labandeira, C. C.,
Pandolfi, J. M., Wing, S. L., \& Bobe, R. (2004). Long-term stasis in
ecological assemblages: Evidence from the fossil record. \emph{Annu.
Rev. Ecol. Evol. Syst.}, \emph{35}(1), 285--322.

\bibitem[\citeproctext]{ref-erwin2009climate}
Erwin, D. H. (2009). Climate as a driver of evolutionary change.
\emph{Current Biology}, \emph{19}(14), R575--R583.

\bibitem[\citeproctext]{ref-faith2013climate}
Faith, J. T., \& Behrensmeyer, A. K. (2013). Climate change and faunal
turnover: Testing the mechanics of the turnover-pulse hypothesis with
south african fossil data. \emph{Paleobiology}, \emph{39}(4), 609--627.

\bibitem[\citeproctext]{ref-de2022predicted}
Fuente, A. de la, Krockenberger, A., Hirsch, B., Cernusak, L., \&
Williams, S. E. (2022). Predicted alteration of vertebrate communities
in response to climate-induced elevational shifts. \emph{Diversity and
Distributions}, \emph{28}(6), 1180--1190.

\bibitem[\citeproctext]{ref-fukami2015historical}
Fukami, T. (2015). Historical contingency in community assembly:
Integrating niches, species pools, and priority effects. \emph{Annual
Review of Ecology, Evolution, and Systematics}, \emph{46}(1), 1--23.

\bibitem[\citeproctext]{ref-gerhold2018deep}
Gerhold, P., Carlucci, M. B., Procheş, Ş., \& Prinzing, A. (2018). The
deep past controls the phylogenetic structure of present, local
communities. \emph{Annual Review of Ecology, Evolution, and
Systematics}, \emph{49}, 477--497.

\bibitem[\citeproctext]{ref-hagen2023coupling}
Hagen, O. (2023). Coupling eco-evolutionary mechanisms with deep-time
environmental dynamics to understand biodiversity patterns.
\emph{Ecography}, \emph{2023}(4), e06132.

\bibitem[\citeproctext]{ref-hargreaves2009importance}
Hargreaves, J., \& Annan, J. (2009). On the importance of paleoclimate
modelling for improving predictions of future climate change.
\emph{Climate of the Past}, \emph{5}(4), 803--814.

\bibitem[\citeproctext]{ref-harrison2020climate}
Harrison, S., Spasojevic, M. J., \& Li, D. (2020). Climate and plant
community diversity in space and time. \emph{Proceedings of the National
Academy of Sciences}, \emph{117}(9), 4464--4470.

\bibitem[\citeproctext]{ref-jabot2020assessing}
Jabot, F., Laroche, F., Massol, F., Arthaud, F., Crabot, J., Dubart, M.,
et al. (2020). Assessing metacommunity processes through signatures in
spatiotemporal turnover of community composition. \emph{Ecology
Letters}, \emph{23}(9), 1330--1339.

\bibitem[\citeproctext]{ref-jackson2006can}
Jackson, J. B., \& Erwin, D. H. (2006). What can we learn about ecology
and evolution from the fossil record? \emph{Trends in Ecology \&
Evolution}, \emph{21}(6), 322--328.

\bibitem[\citeproctext]{ref-jackson2015community}
Jackson, S. T., \& Blois, J. L. (2015). Community ecology in a changing
environment: Perspectives from the quaternary. \emph{Proceedings of the
National Academy of Sciences}, \emph{112}(16), 4915--4921.

\bibitem[\citeproctext]{ref-Lewis2023paleo}
Jones, L. A., Gearty, W., Allen, B. J., Eichenseer, K., Dean, C. D.,
Galván, S., et al. (2023). Palaeoverse: A community-driven r package to
support palaeobiological analysis. \emph{Methods in Ecology and
Evolution}, 1--11. \url{https://doi.org/10.1111/2041-210X.14099}

\bibitem[\citeproctext]{ref-khattar2021determinism}
Khattar, G., Macedo, M., Monteiro, R., \& Peres-Neto, P. (2021).
Determinism and stochasticity in the spatial--temporal continuum of
ecological communities: The case of tropical mountains.
\emph{Ecography}, \emph{44}(9), 1391--1402.

\bibitem[\citeproctext]{ref-kiessling2023improving}
Kiessling, W., Smith, J. A., \& Raja, N. B. (2023). Improving the
relevance of paleontology to climate change policy. \emph{Proceedings of
the National Academy of Sciences}, \emph{120}(7), e2201926119.

\bibitem[\citeproctext]{ref-kraft2015community}
Kraft, N. J., Adler, P. B., Godoy, O., James, E. C., Fuller, S., \&
Levine, J. M. (2015). Community assembly, coexistence and the
environmental filtering metaphor. \emph{Functional Ecology},
\emph{29}(5), 592--599.

\bibitem[\citeproctext]{ref-ludwig2019perspectives}
Ludwig, P., Gómez-Navarro, J. J., Pinto, J. G., Raible, C. C., Wagner,
S., \& Zorita, E. (2019). Perspectives of regional paleoclimate
modeling. \emph{Annals of the New York Academy of Sciences},
\emph{1436}(1), 54--69.

\bibitem[\citeproctext]{ref-merdith2021extending}
Merdith, A. S., Williams, S. E., Collins, A. S., Tetley, M. G., Mulder,
J. A., Blades, M. L., et al. (2021). Extending full-plate tectonic
models into deep time: Linking the neoproterozoic and the phanerozoic.
\emph{Earth-Science Reviews}, \emph{214}, 103477.

\bibitem[\citeproctext]{ref-mottl2021global}
Mottl, O., Flantua, S. G., Bhatta, K. P., Felde, V. A., Giesecke, T.,
Goring, S., et al. (2021). Global acceleration in rates of vegetation
change over the past 18,000 years. \emph{Science}, \emph{372}(6544),
860--864.

\bibitem[\citeproctext]{ref-myers2015paleoenm}
Myers, C. E., Stigall, A. L., \& Lieberman, B. S. (2015). PaleoENM:
Applying ecological niche modeling to the fossil record.
\emph{Paleobiology}, \emph{41}(2), 226--244.

\bibitem[\citeproctext]{ref-nogues2009predicting}
Nogués-Bravo, D. (2009). Predicting the past distribution of species
climatic niches. \emph{Global Ecology and Biogeography}, \emph{18}(5),
521--531.

\bibitem[\citeproctext]{ref-nogues2018cracking}
Nogués-Bravo, D., Rodrı́guez-Sánchez, F., Orsini, L., Boer, E. de,
Jansson, R., Morlon, H., et al. (2018). Cracking the code of
biodiversity responses to past climate change. \emph{Trends in Ecology
\& Evolution}, \emph{33}(10), 765--776.

\bibitem[\citeproctext]{ref-sonne2022biodiversity}
Sonne, J., Dalsgaard, B., Borregaard, M. K., Kennedy, J., Fjeldså, J.,
\& Rahbek, C. (2022). Biodiversity cradles and museums segregating
within hotspots of endemism. \emph{Proceedings of the Royal Society B},
\emph{289}(1981), 20221102.

\bibitem[\citeproctext]{ref-vasconcelos2022retiring}
Vasconcelos, T., O'Meara, B. C., \& Beaulieu, J. M. (2022). Retiring
{``cradles''} and {``museums''} of biodiversity. \emph{The American
Naturalist}, \emph{199}(2), 194--205.

\bibitem[\citeproctext]{ref-weiher2011advances}
Weiher, E., Freund, D., Bunton, T., Stefanski, A., Lee, T., \&
Bentivenga, S. (2011). Advances, challenges and a developing synthesis
of ecological community assembly theory. \emph{Philosophical
Transactions of the Royal Society B: Biological Sciences},
\emph{366}(1576), 2403--2413.

\bibitem[\citeproctext]{ref-zachos2001trends}
Zachos, J., Pagani, M., Sloan, L., Thomas, E., \& Billups, K. (2001).
Trends, rhythms, and aberrations in global climate 65 ma to present.
\emph{Science}, \emph{292}(5517), 686--693.

\end{CSLReferences}




\end{document}
